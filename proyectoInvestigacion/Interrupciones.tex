\documentclass[]{article}
\usepackage{multicol}
\bibliography{int_references}
%opening
\title{Investigación: Interrupciones en microprocesadores}
\author{Cruz Buitrago, Edgar Julian \\ julian.cruz1@udea.edu.co \\ Facultad de Ingeniería \\ Universidad de Antioquia \\ Medellín, Antioquia}
\date{Julio 2020}
\begin{document}

\maketitle
\begin{multicols}{2}
	\begin{abstract}		
		
	\end{abstract}
	
	\section{Definición}
		Una interrupción es una señal que hace que un microprocesador trabaje temporalmente en una tarea diferente y luego vuelva a su tarea que tenía en curso.\cite{wikibooks}  Dicha señal es conocida como una solicitud de interrupción o IRQ por sus siglas en inglés; y son procesadas por un controlador de interrupciones, también llamado rutina de servicio de interrupción, o ISR por sus siglas en inglés.\cite{techterms}  Las interrupciones son de gran utilidad para dar atención a situaciones que requieren un tratamiento diferente o que inciden en el comportamiento de un sistema.
	\section{Antecedentes}
		Consultando la historia de las interrupciones nos encontramos con dos sistemas que son referenciados como los primeros en utilizarlas; el primero de ellos la UNIVAC 1103 (1953) el cual es citado en el libro de Bell y Newell, en una linea de tiempo de la computación en la categoría de procesamiento concurrente\cite{history2}.  A pesar de que en el manual de la UNIVAC 1103\cite{history3} no se describen las interrupciones, publicaciones posteriores realizadas por los laboratorios que tuvieron acceso a ella dan evidencia de implementaciones para múltiples dispositivos de entrada y salida; sin embargo otros autores citan a la UNIVAC I (1951) como el primer sistema en implementar la interrupciones ya que contaba con un manejo de excepciones para el desbordamiento aritmético.  Más adelante aparece el primer uso de enmascaramiento de interrupción en la IBM 650 (1954), las primeras interrupciones de entrada y salida en la NBS DYSEAC (1954) y la captura de saltos o transferencias como fue llamado en su época en la IBM 704 (1955).\cite{history}
	\section{Clasificación}
		
	\section{Aplicación en Software}
	\section{Ejemplo}
\end{multicols}

\end{document}
