\documentclass[]{article}
\usepackage[utf8]{inputenc}
\usepackage[spanish]{babel}
\usepackage{multicol}
\usepackage{biblatex}
\usepackage{ragged2e} 
\addbibresource{hil_references.bib}
\usepackage{geometry}
\geometry{
	letterpaper,
	left=25mm,
	right=20mm,
	top=25mm,
	bottom=20mm,
}
%opening
\title{Investigación: Hilos en microprocesadores}
\author{Cruz Buitrago, Edgar Julian \\ julian.cruz1@udea.edu.co \\ Facultad de Ingeniería \\ Universidad de Antioquia \\ Medellín, Antioquia}
\date{Julio 2020}
\begin{document}
	
	\maketitle
	\begin{multicols}{2}
		\justify
		\begin{abstract}		
			La ejecución en paralelo fue una de las funcionalidades que rápidamente la evolución de las computadoras fue exigiendo para mejorar el rendimiento en la operación de las mismas; múltiples soluciones se han implementado para solucionarlo entre ellas el procesamiento multihilo.
		\end{abstract}
		{\smallskip \keywords Palabras clave: c++, qt, hilos, microprocesadores.}
		\section{Definición}
		Un hilo puede ser definido como el flujo de instrucciones para realizar una tarea.\cite{jesgargardon}  Los hilos son ejecutados por los procesos y se pueden ejecutar de manera individual mientras comparten recursos; pero ejecutar varios hilos dependerá de si el proceso fue diseñado de esa manera. En realidad, las tareas no se ejecutan simultáneamente, el núcleo va alternando las instrucciones que ejecuta de uno y otro hilo, y como lo hace a gran velocidad, tenemos la sensación de que ambos hilos se ejecutan a la vez, que ambas tareas se están ejecutando al mismo tiempo, aunque en realidad no es así.\cite{whatsabyte} Los hilos son de gran utilidad para reducir los tiempos de proceso y eficientar el rendimiento de un programa.
		\section{Antecedentes}
		La evolución de las computadoras mantiene una constante búsqueda de mejorar el rendimiento de las mismas, al principio apoyado por la aparición de transistores más rápidos y mas pequeños, se opto por técnicas como: a) Aumentar el ancho de banda de los datos que se procesan. b) Utilizar circuitos combinatorios rápidos, como sumadores o multiplicadores. c) Crear archivos de registro más grande. d) Aumentar la memoria cache en los chip; pero todo esto aplicado para un solo procesador fue evidenciando algunas limitantes en la velocidad del reloj, porque a mas velocidad de reloj, es requerido mas potencia y aparecen problemas de disipación de calor. Fue por esto que empezó a abordarse los conceptos de paralelismo, donde múltiples subprocesos se ejecuten en paralelo y se empiezan a hablar de multiples hilos de ejecución. \cite{history2} En el articulo "Historia de multihilos" de Mark Smotherman\cite{history1} nos muestra una linea de tiempo acerca del uso hilos en los microprocesadores donde vemos que aparecen desde la década de los 50 y cada vez mas capacidades según las capacidades de la electrónica mejoraban. 
		\section{Clasificación}
		Existen las siguientes categorías en la implementación de los hilos:\\
		Hilos a nivel de núcleo (KLT por sus siglas en ingles Kernel Level Thread) son creados por el usuario en una equivalencia 1 a 1 con entidades de procesamiento en el kernel.\\
		Hilos a nivel de usuario (ULT por sus siglas en ingles User Level Thread) este modelo implica que toda implementación de hilos a nivel de aplicación es mapeada a una entidad de procesamiento en el kernel. El kernel no tiene conocimiento del número de hilos en la aplicación. Tiene algunas desventajas a nivel de priorización de tareas.\\
		Implementaciones híbridas, se dan cuando se combinan las dos técnicas anteriores y aumentan la capacidad de procesamiento al poder procesar múltiples hilos de aplicación en múltiples hilos del kernel.\cite{wiki}.\\
		\section{Aplicación en Software}
		Las implementaciones multihilo son de gran utilidad para mejorar el rendimiento de las aplicaciones de software y dividir las tareas requeridas en tiempo de ejecución. Según el lenguaje de programación el uso de esta característica se puede facilitar o incluso ni siquiera necesitar por la naturaleza del lenguaje.
		
		\printbibliography
	\end{multicols}
	
\end{document}