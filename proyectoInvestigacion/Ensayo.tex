\documentclass{article}
\usepackage[utf8]{inputenc}
\usepackage[spanish]{babel}
\usepackage{multicol}
\usepackage{ragged2e} 
\usepackage{biblatex}
\addbibresource{referencias.bib}
\usepackage{geometry}
 \geometry{
 letterpaper,
 left=25mm,
 right=20mm,
 top=25mm,
 bottom=20mm,
 }

\title{Ensayo: La crisis de los fundamentos y el nacimiento de la computación}
\author{Cruz Buitrago, Edgar Julian \\ Facultad de Ingeniería \\ 
        Universidad de Antioquia \\ Medellín, Colombia \\
        julian.cruz1@udea.edu.co}
\date{Marzo 2020}

\begin{document}
\maketitle
\begin{multicols}{2}
    \justify
        \textbf{\textit{Resumen} - El nacimiento de la computación moderna tuvo grandes discusiones, diferencias y controversias; las cuales siempre permitieron un crecimiento en interesados en demostrar y aportar a descubrir las bases de esta especialidad.}
        \textbf{\\ \textit{Palabras clave} - Gödel, Turing, Hilbert, Maquina Universal, Computación, Codificación, Maquina de Turing.} 
        
        \vspace{\baselineskip}
    
        Muchas son las afirmaciones con relación a la máquina de Turing como pilar en el nacimiento de la computación moderna, pero antes siquiera que Turing tuviera la idea en mente hubo una serie de sucesos y personajes que permitieron tal logro. A principios del siglo XX, fue publicada la obra Principia Mathematica escrita por Alfred North Whitehead y Bertrand Russell, y publicada por primera vez en tres volúmenes en 1910, 1912 y 1913. Si bien la obra era un testimonio del gran protagonismo que tuvieron las ciencias en el siglo XIX, el libro fue base para el desarrollo y popularización de la lógica matemática moderna y un gran impulso para la investigación en los fundamentos de las matemáticas a lo largo del siglo XX.\cite{principia}
        
        \vspace{\baselineskip}
        
        Uno de los frentes en esas investigaciones fue el tomado por el matemático alemán David Hilbert, que se esmeró en demostrar el carácter \textbf{completo, consistente y finitario} de las matemáticas; pero otro justo enfocado en refutar dicha afirmación.\cite{bbva}  Fue en 1931 en un artículo titulado ''Sobre las proposiciones formalmente irresolubles de Principia Mathematica y sistemas emparentados'' donde el matemático Kurt Gödel demostró que las matemáticas, tal y como las conocemos, no pueden ser utilizadas para demostrar el carácter consistente o completo de las mismas.
        
        \vspace{\baselineskip}
        
        El método de Gödel se basaba en que toda oración aritmética era posible expresarla de forma numérica, esto daba una {codificación} a los operandos donde, por ejemplo, la oración \(1 + 1 = 2\) se reescribía primero de esta forma \(s0 + s0 = ss0\) y luego estos signos se reescribían con sus equivalentes numéricos y se finalizaba con una multiplicación entre si de números primos sucesivos elevados a las potencias de los números antes indicados.
        \[1 + 1 = 2\]
        \[s0 + s0 = ss0\]
        \[76 11 76 5 776\]
        \[2^7 * 3^6 * 5^11 * 7^7 * 11^6 * 13^5 * 17^7 * 19^7 * 23^6 = ?\]
        
        \vspace{\baselineskip}
        
        El método se basaba en el teorema fundamental de las matemáticas que declara que cualquier número entero positivo puede ser representado sólo de una manera como producto de números primos.  Si bien era un cálculo donde haría falta una computadora, para ese entonces Gödel no tenía una computadora en mente; lo que buscaba era tener la forma de codificar afirmaciones matemáticas y en específico codificar aquella frase que derrumbara todas la murallas, citaba así, ''La fórmula G, para la que el número Gödel es g, declara que existe una formula con el numero Gödel g que no es demostrable en Principia Mathematica o cualquier sistema emparentado''.  Con esta fórmula Gödel postulaba el carácter indemostrable de la misma. Si semejante formula es cierta, no es demostrable; si es demostrable no es cierta. No obstante, en un sistema \textbf{completo}, uno debería poder probar o refutar todas las afirmaciones hechas utilizando ese sistema, en tanto que, en un sistema matemático \textbf{consistente}, debería resultar imposible demostrar una afirmación que no fuera cierta o refutar una afirmación cierta. Gödel acababa de hacer ambas cosas.\cite{turing}
        
        \vspace{\baselineskip}
        
        Tras este golpe, uno de los trabajos de Hilbert, ''El problema de la decisión'' quedaba sin demostración o refutación consistente; y fue justo el problema en el que Alan Turing se propuso trabajar, la primera vez que se topó con este problema fue en 1934 en una clase con Maxwell Newman sobre los fundamentos de las matemáticas y a partir de allí trabajo incansablemente para finalmente presentar sus resultados en un artículo titulado {Números computables, con una aplicación al problema de la desición} publicado a comienzos de 1937 en "\textit{Proceedings of the London Mathematical Society}". Este artículo está dividido en tres secciones, la primera parte introduce los ''número computable'' y ''máquina computadora''; la segunda postula el concepto de una ''máquina universal'' y la tercera emplea esos conceptos para demostrar que el problema de la decisión es insoluble.  Fue un artículo que para su época generó bastante revuelo, solo con el simple hecho de hablar de una hipotética ''máquina'' computadora en un artículo matemático, era quebrantar leyes de una ortodoxia muy rígida; incluso para ese tiempo se llamaba ''computadora'' a aquellas personas que realizaban los cálculos, en su mayoría mujeres.\cite{turing}
        
        \vspace{\baselineskip}
        
        Es gratificante notar en esta breve historia, un factor común en la personalidad de los científicos, su deseo inquietante de demostrar, como diríamos hoy en día, de no dar por hecho sino de comprobar lo expuesto; tal como nos dice el profesor del curso INFORMÁTICA II, {no me crean a mí, consúltenlo o compruébenlo} y mucho más inspirador saber que un científico de tan solo 24 años lograra fundar las bases de la computación moderna.  Opino que esta característica debe acompañarnos en nuestra vida académica y profesional como ingenieros, y aprovechar al máximo la infinidad de información y herramientas a las que tenemos acceso para no dejar diezmar nuestro espíritu investigativo.
        
        \printbibliography
    
\end{multicols}

\end{document}
